\section{Introduction}

ESports, also known as electronic sports, has become an official sport since the 21st century.
The new technologies applied in the eSports are attractive to the young and old.
The 3D graphic display, vivid sound effect, immersive experience, and the interesting operations make eSports more and more popular.
According to a report~\cite{esports}, In 2013, it was estimated that approximately 71.5 million people worldwide watched eSports.

Since eSports games are increasingly popular, the prediction of eSports game result has become increasingly significant.
The estimation of game results can be used to train the professional players at ordinary times and modify the gamer’s tactic for the upcoming game.
This estimation can also be used for media.
In addition, the predictions of the game results can improve the entertainment for fans.

Therefore, we decide to design and implement a prediction system based on machine learning technology for eSports games.
In this paper, we focus on the one of the most popular eSports games is Dota.
Dota is the short name for Defense of the Ancients.
Dota~\cite{dotablog} is a free-to-play multiplayer online battle arena video game.
Two five-player teams compete in the playing field, where each player chooses one from a total of 111 heroes to play.
Millions of game players take part in Dota every day.
With the highly-speed development of eSports, abundant and grand eSports competitions are organized globally every month, season, and year.
Prize pool of millions of dollars is often provided by some competitions.

Prior research 
Conley et al.~\cite{conley2013does} did a research on which machine learning method is the best for data game results.
However, they only focus on two machine learning approaches, linear regression and k-nearest neighbour(kNN).
Kalyannaraman~\cite{kau2013win} did an augmented algorithm on traditional linear regression considering pair relationships of heroes.
However, he did not consider all the relationship between heroes, like the restraint relationship between heroes.
What's more, these two research did not test on large scale of data, which means not so reliable on their results.
Drachen et al. ~\cite{drachen2014skill} try to predict the temporal and spatial of players in the game based on the hero skills.
But the objective is different from ours. 

Our goal is to predict the results of Dota Game.
It means, given the specific two groups of heroes, we can give back the users the predicted outcome of the game.
In this paper, the contributions include:

\begin{itemize}
\item We are the first one to consider the restraint relationship of heroes into the prediciton of Dota game result. 
\item We are the first one to introduce the game lasting time as a feature vector in the prediction. 
\item We abstract the Dota game prediction problem and applied different machine learning approaches.
\item Through tuning the parameters in the machine learning approaches, we compare the performance among different algorithms. We also conclude the results and analyze the results.
\end{itemize}

The rest of the paper proceeds as follows.
Section 2 introduces Dota and mainstream machine learning algorithms we try in this paper.
Section 3 gives a brief introduction about the abstract of the problem and the goal.
Then in the section 4, we talk about the design of our machine learning algorithms; also a detailed description of Decision Tree is given.
In the section 5, we will describe our data sets, training and testing methods, evaluation metrics and results.
Section 6 introduces some related papers.
In the final section, we will analyze the results of the prediction.
