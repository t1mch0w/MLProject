\section{Conclusion}
This paper proposes a method to predict the eSports game results of Dota.
With three-type features (Team ID, Hero Lineup and Result) and decision tree model, our Dota prediction system gets 65.4\% F-score from a data set of 15,197 matches.
 
The F-score is not relatively high and the reasons are as follows:
\begin{enumerate}
\item Concerning the data sets, we can only get the game records including the information of Team ID, Heroes Lineup and result.
More detailed information about teams and games are not available.

\item Regarding features, player’s styles and skills, physical and emotional condition and the external environment all are vital to the result of a game, but none of them can be measured.

\item DotA2 is a game of uncertainty with regards to the in game mechanism. Many elements in the map are generated randomly, such as type of runes, and Roshan Respouse time. Like poker games, luck can sometimes play an important role in determining the victory team of a match.

\end{enumerate}

